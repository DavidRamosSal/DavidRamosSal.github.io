\begin{rubric}{Perfil Profesional}
\text{Soy un físico con más de un año de experiencia en investigación en física aplicada, tutorías universitarias en física y cálculo, y la creación de material pedagógico para la concientización sobre el cambio climático. Como físico, mi formación en matemáticas universitarias es sólida y hago uso de ellas de manera cotidiana en mi trabajo de investigación. Mi experiencia en tutorías y proyectos educativos me ha permitido aprender a expresar ideas complejas de una manera accesesible a audiencias con distintos niveles de conocimiento técnico. Manejo además distintos tipos de software de cálculo simbólico que, en mi opinión, tienen un gran potencial para complementar el aprendizaje de matemáticas universitarias al permitir visualizar resultados y explorar problemas complejos.
%Soy un físico bilingüe finalizando un programa de maestría en física. Tengo experiencia en investigación en física aplicada, se plantear problemas de investigación, explorar la literatura acerca del tema de interés, construir y validar modelos basado en el análisis de datos y comunicar resultados efectivamente. Además, tengo una corta experiencia en la creación de material didáctico para la educación sobre el cambio climático donde aprendí los principios de la ciencia del cambio climático y adquirí un interés por participar en proyectos de innovación tecnológica para afrontar los retos del cambio climático. 
%Quiero decir: 
%soy versátil adaptable a distintas disciplinas,
%Mis conocimientos de física y habilidades computacionales me han permitido participar en proyectos de variada naturaleza.
%capaz de aprender y realizar investigación en un nuevo tema así como capaz de comunicarme y escuchar a otros,
%interesado de entrar a la insdustria energética y seguro de que mi experiencia es trasladable.
%Mi formación académica me ha permitido participar en proyectos teóricos, prácticos y educativos (alguna forma de hablar de mi versatilidad). Y mi experiencia en investigación me ha enseñado a comunicar resultados (capacidad de comunicación) y 
}
% Blank lines result in extra space!
%
%\entry*[2013 -- 2015]%
%	\textbf{Lecturer.} Information Technology Department, School of Engineering, Science and Technology, XYZ College.
%
\end{rubric}